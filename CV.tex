\documentclass[12pt,a4paper]{res}
\usepackage{a4-mancs}
\usepackage{pslatex}
\usepackage{fancyhdr}
\usepackage{multicol}
\setlength\columnsep{40pt}
\setlength{\topmargin}{-0.6in}
\setlength{\textheight}{9.8in}
\setlength{\headsep}{0.2in}
\setlength{\headheight}{14.49998pt}
\setlength{\textheight}{297mm}
\renewcommand{\headrulewidth}{0pt}
\cfoot{}
\pagestyle{fancy}

\begin{document}
\thispagestyle{empty}
  \begin{center}
  \Large{Dragos Romario Rosu  \\ [12pt]}
  \normalsize 96 Jackson Crescent, Manchester, M15 5AA\\
  Mobile: 0774 340 7461, Email: r.dragosh@yahoo.com\\
  Github: github.com/rosudrag
  \end{center}
  
\begin{resume}
\vspace{-10mm}
\section{\large\bf Professional Experience}

\section{\bf 2015 - Present \hspace{0.5mm} Code ComputerLove - Software Developer:}
\vspace{5mm}    
	\begin{itemize}
	\end{itemize}
\section{\bf 2013 - 2014 \hspace{0.5mm} Industrial Placement - CIVICA - Software Developer:}
\vspace{5mm}    
	\begin{itemize}
	\item[] Put in practice the techniques taught at University to quickly grasp the technologies the company used and made myself a real asset for the team.
	\item{\bf Nominated for the best team in CIVICA:} The team I was working in came second place at best team awards in the company and I would like to think I played an important role in this.
	\item{\bf Skills and knowledge acquired:} C\# and OracleSQL and other specific Microsoft technologies: WCF, EntityFramework, ServiceBus, WPF. Other skills: SCRUM Agile development and pair programming.
	\end{itemize}


\section{\large\bf Education}
\vspace{5mm}

\section{\bf 2011 - 2015 \hspace{1.5mm}The University of Manchester}
  
  \begin{itemize} % \item[] to prevent a bullet from appearing
     \item[] BSc (Hons) Computer Science  with Industrial Experience
     \item[] Expected result 2:1 \hspace{10mm}Second year results: 65\% 

     \item Developed an excellent IT foundation by combining the skills and knowledge taught
     at University with practical experience.
     \item Regularly worked in small size teams to plan, design and develop projects,   employing techniques such as agile and pair programming.
     \item Further improved my algorithmic thinking by participating in challenging puzzle solving events and competitions.
     \item Acquired the ability to view a problem through different perspectives: the client, developer and investor view and also the skill to combine them in order to produce a solution that will benefit all the involved parties.
   \end{itemize}

\section{\bf Notable courses and themes:}
\vspace{2mm}
	\begin{itemize}		
	      \begin{multicols}{2}
		\item \vspace{-1.5mm} Advanced Algorithms
		\item \vspace{-1.5mm} Software Engineering and Patterns
		\item \vspace{-1.5mm} Computer Architecture
		\item \vspace{-1.5mm} Computer Networks
		\item \vspace{-1.5mm} Computer Graphics and Image Processing
		\item \vspace{-1.5mm} Databases
		\item \vspace{-1.5mm} Human Computer Interaction
		\item \vspace{-1.5mm} Logic and Modelling, Verified Development
		\item \vspace{-1.5mm} Documents and the Web
		\item \vspace{-1.5mm} Distributed Systems and Mobile Systems
	      \end{multicols}
	\end{itemize}
      
\section{\bf Manchester Ultimate Programming and achievement:}
\vspace{5mm}    
    \begin{itemize}
    \item[] Conducted around 20 presentations. Organised one 24 hour hackathon where I managed to put my networking skills into practice to acquire sponsorship for prizes and catering.
    \item{\bf Most Innovative technical approach award:} Awarded at the "Open Source Hackathon". Teamed with a PhD student to create a project called "Group My contacts". I suggested using pair programming with the Cloud9 IDE and git on the separate tasks. Analysed various issues that arose when trying to bind the work together and quickly pinpointed them out and devised a common solution.

      \item {\bf Leadership:} Through hard work and dedication I worked my way up the hierarchy of the society and became the chairman of it. The events turnout had doubled as result.
     \end{itemize}
\section{\large\bf Skills}
\vspace{5mm}
\end{resume}
\end{document}